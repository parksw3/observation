\documentclass{article}\usepackage[]{graphicx}\usepackage[]{color}

\title{A note on observational processes in epidemic models}
\author{Sang Woo Park and Benjamin M. Bolker}

\usepackage{tabularx}

\usepackage{amsmath}
\usepackage{natbib}
\usepackage{hyperref}
\bibliographystyle{chicago}
\date{\today}

\usepackage{bm}

\usepackage{afterpage}
\usepackage{pdflscape}

\newcommand{\etal}{\textit{et al.}}

\newcommand{\comment}[3]{\textcolor{#1}{\textbf{[#2: }\textit{#3}\textbf{]}}}
\newcommand{\bmb}[1]{\comment{cyan}{BMB}{#1}}
\newcommand{\swp}[1]{\comment{magenta}{SWP}{#1}}
\newcommand{\citetapos}[1]{\citeauthor{#1}'s \citeyearpar{#1}}

\newcommand{\fref}[1]{Fig.~\ref{fig:#1}}
\begin{document}

\maketitle

\section{Introduction}

Mechanistic analyses of epidemic time series allow us to infer the underlying 
transmission mechanism, estimate biologically relevant parameters, and 
predict the course of an epidemic \citep{breto2009time}. In order to make a 
precise and accurate inference, disease modelers often focus on developing
more realistic process models. For example, the same London measles time 
series from the prevaccination era has been analyzed using multiple models -- 
these models account for time-varying transmission rates \citep{fine1982measles}, 
realistic age structure \citep{schenzle1984age},
metapopulation structure \citep{xia2004measles}, continuous-time infection process 
\citep{cauchemez2008likelihood}, and extra-demographic variability \citep{he2009plug}. 

Despite the amount of effort put into developing \emph{better} process models, disease 
modelers often neglect details of the observation process associated with disease
case reports. Most disease models assume that new cases are reported
instantaneously when an individual is infected (e.g., \cite{martinez2016differential, 
kennedy2018modeling, pons2018serotype}) or develops symptom; one exception is
the model by \citep{he2009plug}, which assumes that infections are counted upon
recovery (because diagnosed cases are put to bed rest and are effectively 
no longer infectious). 

Here, we use a simple Susceptible-Infected-Removed (SIR) model to study how 
assumptions about the underlying observation process affect parameter estimates
of the SIR model. We show that making wrong assumptions about the timing of 
incidence reports can lead to biases in parameter estimates and narrow 
confidence intervals. 

\section{Methods}

The Susceptible-Infected-Removed (SIR) model describes how disease spreads in a
homogeneous population:
\begin{equation}
\begin{aligned}
\frac{dS}{dt} &= - \beta S \frac{I}{N}\\
\frac{dI}{dt} &= \beta S \frac{I}{N} - \gamma I\\
\frac{dR}{dt} &= \gamma I,
\end{aligned}
\end{equation}
where $\beta$ is the contact rate, $\gamma$ is the recovery rate, and $N = S + I + R$ is
the total population size. We define incidence at time $t$ as the number of newly infected
individuals that are infected between time $t- \Delta t$ and time $t$, where $\Delta t$ is
the reporting time step. We expect infected individuals to report their infection some time
after their infection; the number of reported cases defines the \emph{observed} incidence. 

For brevity, we consider two extreme cases: individuals instantaneously report
their infection when they become infected or when they recover. The observed incidence 
measured upon infection ($i_1(t)$) can be defined by the integral:
\begin{equation}
i_1(t) = \int_{t - \Delta t}^{t} \beta S \frac{I}{N} dt,
\end{equation}
where $\Delta t$ is the reporting time step.
Alternatively, we can keep track of cumulative incidence, $C$, by adding a 
new state variable described by $dC/dt = \beta S I/N$ and taking the difference between 
the two consecutive reporting periods: $i_1(t) = C(t) - C(t-\Delta t)$. Likewise, 
incidence measured upon recovery ($i_2(t)$) can be defined by the integral:
\begin{equation}
i_2(t) = \int_{t-\Delta t}^{t} \gamma I dt,
\end{equation}
or by the consecutive difference in the cumulative number of recovered cases:
$i_2(t) = R(t) - R(t - \Delta t)$.
Finally, we model under-reporting using a negative binomial distribution with
mean $\rho i_1(t)$ or $\rho i_2(t)$, where $\rho$ is the reporting rate, and
an over dispersion parameter $\theta$. For convenience, we will refer to these two
negative binomial models as infection model and recovery model hereafter; 
likewise, we will refer to epidemic time series generated from these models 
as infection time series and recovery time series.

In this study, we focus on estimating 5 parameters: the basic reproductive
number $\mathcal R_0 = \beta/\gamma$, mean infectious period $1/\gamma$, 
reporting rate $\rho$, an over dispersion parameter $\theta$ and the initial
proportion of the infected individuals $i_0$. Initial proportion of
susceptible individuals is assumed to be $1 - i_0$. The total population
size $N$ is assumed to be known.

\section{Results}

In order to introduce the problem, we first compare the dynamics of observed 
incidence, $i_1(t)$ and $i_2(t)$, (Figure 1). Lags in reporting time delay 
the observed epidemic peak timing and reduce the size of the peak. However, 
these differences in the reporting time have small effect to the overall shape 
of the epidemic curve. In the presence of observation and process error, we 
would not expect to be able to distinguish between the two reporting processes. 

Note that incidence measured at these two periods are different from prevalence,
$I(t)$, which is defined as the number of \emph{currently} infected individuals;
the dynamics of observed incidence depend on the reporting time step (because the sum of 
true incidence is equal to the final size of an epidemic), whereas those of
prevalence do not. We expect the dynamics of incidence and prevalence to be similar only when
the reporting time step is equal to the disease generation time. While it is
uncommon, some models do not make a clear distinction between the two.

\begin{figure}
\includegraphics[width=\textwidth]{../figure/example.pdf}
\caption{
\textbf{A comparison of incidence measured at two different time points in infection.}
We used the following parameters to simulate the SIR model and generate the figure: 
$\mathcal R_0 = 2$, $1/\gamma = 1$ time units, $N = 1 \times 10^5$, $i0 = 1 \times 10^{-4}$,
and $\Delta t = 0.1$ time units.
}
\end{figure}

In order to understand how assumptions about the timing of case reporting affect 
parameter estimates of the SIR model, we simulate observed epidemic curves 
(i.e., infection time series and recovery time series), measured upon both infection and
recovery, 100 times and fit both models (i.e., infection model and recovery model) to each 
time series. We compare the estimates of the basic reproductive number $\mathcal R_0$,
and its coverage probability, defined as the proportion of confidence intervals that
contain the true value (95\% confidence interval is expected to contain the true value
95\% of the time by definition). We summarize the results in Figure 2.

When we try to estimate all 5 parameters, fitting the recovery model to
infection time series underestimates the basic reproduction number and 
gives a slightly low coverage (Figure 2A). Fitting the infection model to recovery
time series slightly overestimates the basic reproduction number but gives 
good coverage. Fitting the correct models to their corresponding time series
gives unbiased estimates and good coverage.

\begin{figure}
\includegraphics[width=\textwidth]{../figure/compare_deterministic.pdf}
\caption{
\textbf{A comparison of incidence measured at two different time points in infection.}
We used the following parameters to simulate the SIR model: 
$\mathcal R_0 = 2$, $1/\gamma = 1$ time units, $N = 1 \times 10^5$, $i0 = 1 \times 10^{-4}$,
$\theta = 10$, and $\Delta t = 0.1$ time units.
}
\end{figure}

Disease modelers often assume that the mean infectious period of a disease
is known and focus on estimating the basic reproduction number (or the 
transmission rate). When we assume that the mean infectious period is known
and try to estimate the remaining 4 parameters of the SIR model, fitting the
wrong model results in clearer bias and lower coverage (Figure 2B); however, 
bias from fitting the wrong model changes direction (e.g., fitting the infection
model to recovery model overestimates the reproduction number instead). 

Differences in the direction of the bias can be explained by the estimates of the
exponential growth rate ($r = \beta - \gamma$) and the mean infectious period ($1/\gamma$).
When we fit the recovery model to infection time series, the model overestimates
the initial growth rate in order to match the higher epidemic peak of the 
infection time series. When the mean infectious period is fixed, higher growth rate
translates to higher basic reproduction number as we see in Figure 2B. When we allow
the mean infectious period to vary, the model underestimates the mean infectious
period (high $\gamma$), which also decreases the estimate of the basic reproduction number.
Likewise, fitting the infection model to recovery time series overestimates the
mean infectious period (low $\gamma$), increasing the estimate of the basic reproduction number;
when the mean infectious period is fixed, low growth rate estimates (to match the
lower epidemic peak of the recovery time series) decrease the estimate of the
basic reproduction number.

\section{Discussion}

Mathematical modeling of infectious disease outbreaks helps us understand 
how disease spreads in a population. Disease modelers often put a lot of
effort into capturing the transmission mechanism but make naive assumptions
about how cases are reported. Here, we used a simple deterministic SIR model to show that 
making wrong assumptions about observational processes in epidemic models
can give biased estimates of the basic reproduction number and narrow
confidence intervals.

We assumed that newly infected cases are reported instantaneously when
individuals become infected or when they recover. Although neither of these
assumptions are realistic, they are broadly consistent with the assumptions
of other state-of-art epidemic models. More realistic models may distinguish
reported and unreported cases by adding a new state variable or by modeling
an explicit delay kernel in reporting time.

While we have only considered observation processes associated with incidence 
reporting in this study, it is equally as important to consider observation
processes in mortality reporting. In particular, many disease modelers have
tried to infer underlying transmission mechanisms from historical mortality 
data but assumed that deaths are recorded as soon as individuals die. This assumption
does not account for the possibility that delays in reporting time of deaths can
change the shape of an epidemic curve: longer delays are expected to decrease
the size of an epidemic peak and delay the timing of the peak (Figure 1). 
These assumptions may have important effects on their conclusions about the underlying 
transmission mechanism.

Our study shows that seemingly irrelevant changes in the assumptions of an epidemic
model affects the inference of infectious disease transmission.
We caution disease modelers to be more mindful about their decisions in developing
epidemic models and the implications of their model assumptions.

\bibliography{observation}
\end{document}
