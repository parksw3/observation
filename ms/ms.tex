\documentclass{article}\usepackage[]{graphicx}\usepackage[]{color}

\title{Notes on observational processes in epidemic models}
\author{Sang Woo Park and Benjamin M Bolker}

\usepackage{tabularx}

\usepackage{amsmath}
\usepackage{natbib}
\usepackage{hyperref}
\bibliographystyle{chicago}
\date{\today}

\usepackage{bm}

\usepackage{afterpage}
\usepackage{pdflscape}

\newcommand{\etal}{\textit{et al.}}

\newcommand{\comment}[3]{\textcolor{#1}{\textbf{[#2: }\textit{#3}\textbf{]}}}
\newcommand{\bmb}[1]{\comment{cyan}{BMB}{#1}}
\newcommand{\swp}[1]{\comment{magenta}{SWP}{#1}}
\newcommand{\citetapos}[1]{\citeauthor{#1}'s \citeyearpar{#1}}

\newcommand{\fref}[1]{Fig.~\ref{fig:#1}}
\begin{document}

\maketitle

\section{Introduction}

Mechanistic analyses of an epidemic time series allow us to infer the underlying 
transmission mechanism, estimate biologically relevant parameters, and 
predict the course of an epidemic. In order to make a precise and accurate inference, 
disease modelers often focus on developing more realistic process models. For example,
England and Wales measles time series from the prevaccination era have been analyzed
using many different models -- these model account for time-varying transmission
rates \citep{fine1982measles}, realistic age structure \citep{schenzle1984age},
metapopulation structure \citep{xia2004measles}, continuous-time infection process 
\citep{cauchemez2008likelihood}, and extra-demographic variability \citep{he2009plug}. 

Despite the amount of effort put into developing better process models, disease 
modelers often neglect details of the observation process associated with disease
case reports. Most disease models assume that new cases are reported
instantaneously when an individual is infected or shows symptom; one exception is
the model by \citep{he2009plug}, which assumes that infections are counted upon
recovery (because diagnosed cases are put to bed rest and are effectively 
no longer infectious).

Here, we use a simple SIR model to study how observation process affects statistical
inference of underlying parameters. We show that making wrong assumptions about
the observation process can lead to biases in parameter estimates and narrow 
confidence intervals.

\section{Methods}

The Susceptible-Infected-Removed (SIR) model describes how disease spreads in a
homogeneous population:
\begin{equation}
\begin{aligned}
\frac{dS}{dt} &= - \beta S \frac{I}{N}\\
\frac{dI}{dt} &= \beta S \frac{I}{N} - \gamma I\\
\frac{dR}{dt} &= \gamma I,
\end{aligned}
\end{equation}
where $\beta$ is the contact rate, $\gamma$ is the recovery rate, and $N = S + I + R$ is
the total population size. Incidence measured upon infection ($i_1(t)$) can be
defined by the integral:
\begin{equation}
i_1(t) = \int_{t}^{t + \Delta t} \beta S \frac{I}{N} dt,
\end{equation}
where $\Delta t$ is the reporting time step.
Alternatively, we can keep track of cumulative incidence, $C$, by adding a 
new state variable $dC/dt = \beta S I/N$ and take the difference between 
two consecutive reporting periods: $i_1(t) = C(t+\delta t) - C(t)$. Likewise, 
incidence measured upon recovery ($i_2(t)$) can be defined by the integral:
\begin{equation}
i_2(t) = \int_{t}^{t + \Delta t} \gamma I dt,
\end{equation}
or by the difference in cumulative number of recovered cases: $i_2(t) = R(t + \Delta t) - R(t)$.
Finally, we assume that the reported incidence follows a negative binomial 
distribution with mean $\rho i_1(t)$ or $\rho i_2(t)$, where $\rho$ is the 
reporting rate, and a dispersion parameter $\theta$.

\section{Results}

In order to introduce the problem, we first compare the dynamics of $i_1(t)$ and
$i_2(t)$ (Figure ?). Lags in reporting time delay the ``observed'' epidemic peak 
timing and reduce the size of the peak. However, these differences in the reporting
time have small effect to the overall shape of the epidemic curve. In the presence
of observation and process error, we may not expect to be able to distinguish between
the two reporting processes. Note that incidence measured at these two periods are 
different from prevalence, which is represented by the state variable $I(t)$; the
observed dynamics of incidence depend on the reporting time step, whereas those of
prevalence do not (hence we expect incidence and prevalence to be similar only when
the reporting time step is equal to the disease generation time). While it is
uncommon, some models do not make a clear distinction between the two.

Although incidence measured at two different periods are qualitatively very similar,







\bibliography{observation}
\end{document}
